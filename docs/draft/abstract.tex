\begin{abstract} 

Traditionally, the http communication follows the pull model, where the client initiates the request and server responds it. However, for web pages that demand ``instant interaction'' between client/server or client/client, the pull model falls short because the frequent client-initiated communication will incur a huge overhead.

To overcome the problem, people proposed the ``long polling'' (also dubbed as ``comet''). With the long polling technology the http server will not terminate a connection after responded data has been delivered to a client; instead, the server will keep the connection alive so when new event occurs the server can send the updated message to the client immediately.

However this approach requires the server to keep a large amount of active connections. To efficiently manage the connections we proposed to add an event-based reverse proxy server in between of client and backend servers. So when a new client/server session is established, the reverse will (1) keep the the connection open, and (2) retrieves resources on behalf of a client from backend servers. Meanwhile the backend servers can send the updates to the proxy server and the proxy server
will propagate the updates to related active clients.

We plan to benchmark our reverse proxy software with Twitter stream-like applications which will constantly be publishing new feeds to a large number of subscribers. We will compare the performance of the reverse proxy with the traditional opening and closing connection to evaluate the strengths and weaknesses of our approach. 
\end{abstract}
