\section{Future Work}

The current implementation of PushUp Server is a working prototype which illustrates
the ideas of a scalable event driven long polling server. There are several aspects that
wishes to improve in the future:

\begin{itemize}
\item \emph{Beyond the Single-thread Event Loop}: While single-thread event loop 
        significantly simplifies the implementation, it doesn't fully exploit the
        potential of multi-core computers. 
        
        For the next step, we are considering to migrate from twisted
        framework to Speedo library, which is also being developed by our team.\footnote{
            Actually at first our team considered to use Speedo to support event-driven
            development. But after the evaluation of the risks and the workload, we suspended
            the development of the Speedo and switched to ``python + twisted" as the
            development platform.}

        Speedo is also a reactor-style event loop library but supports 
        ``Event Loop + Thread Pool" style development. As Speedo is implemented 
        in the system programming language Go\cite{GoLang}, we are also expecting a 
        significant improvement of the system performance.

\item \emph {Message Exchange Between PushUp Servers}: Currently the publication 
        propagation is achieved by unicasting. This can be a potential bottleneck
        if there were large number of PushUp servers. 

        We may improve this by the ``gossip protocol". A gossip protocol is a 
        style of internode communication protocol that imitates the gossiping 
        in human societies. Gossip protocols are often used in distributed 
        systems to exchange messages without a centralized server. 

        Gossip protocol can also be used to efficiently notify the PushUp servers
        the joining/leaving of other nodes.

\item \emph {Reliablity and Fault Tolerance}:
    Push Up servers are actually inherently replicated, the data written in any
    message queue will be propagated
    to the other servers. This means that as long as there is at least one 
    PushUp server running
    service can still be delivered. In future works, we can tackle the problem of efficiently
    restoring state when a PushUp server restarts.
\end{itemize}
